An athlete training for a triathlon is located at point O(0,0) on the shore of a circular lake of diameter 4 kilometres. To reach the opposite point Q(4,0),

\begin{center}
\begin{tikzpicture}
    % Draw the circle
    \draw[fill=lightblue] (2,0) circle (2);
    % Draw points
    \filldraw[black] (0,0) circle (2pt) node[left] {O};
    \filldraw[black] (2,0) circle (2pt) node[below] {C};
    \filldraw[black] (4,0) circle (2pt) node[right] {Q};
    \filldraw[black] (2,2) circle (2pt) node[above] {P};
    % Draw dashed lines
    \draw[dashed] (0,0) -- (2,0);
    \draw[dashed] (2,0) -- (2,2);
    \draw[dashed] (2,2) -- (4,0);
    % Draw angle
    \draw[->] (2,0) -- (0.5,0);
    \draw[->] (2,0) -- (2,1.5);
    \draw (2.2,0.2) node[right] {$\theta$};
\end{tikzpicture}
\end{center}

(a) Express the coordinates of P in terms of θ.